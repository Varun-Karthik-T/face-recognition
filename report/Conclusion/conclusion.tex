\chapter{\uppercase{Conclusion and Future Work}}
\label{chap:conclusion}
\begin{spacing}{1.5}
\begin{sloppypar}
\section{\uppercase{CONCLUSION}}
The advent of knowledge-driven autonomous driving agents heralds a transformative era in the realm of transportation, signaling the impending reality of driverless cars. This innovative approach leverages the vast potential of machine learning and artificial intelligence to navigate complex environments with a level of precision and adaptability previously unattainable. The significance of this project lies not only in its forward-thinking vision but also in its feasibility within low-resource settings, demonstrating that cutting-edge research and development in autonomous driving technology are not confined to high-end laboratories with abundant funding. It underscores the accessibility of such technologies, paving the way for widespread adoption and iterative improvement.

Furthermore, the project showcases the versatility of Large Language Models (LLMs) as a core component of autonomous driving systems. LLMs' ability to interpret and process vast amounts of data in real-time allows them to make informed decisions, enhancing the safety and reliability of autonomous vehicles. Their expandability ensures that as the technology evolves and new data becomes available, the driving agents can be updated and improved, thus future-proofing the system. The generalizability of LLMs means they can be adapted to various driving conditions and geographies, making them a robust foundation for global deployment.

This project's implications extend beyond the technical sphere, promising to revolutionize how society views mobility and accessibility. By reducing the reliance on human drivers, it opens up new opportunities for individuals who may be unable to drive due to physical limitations or other constraints. It also has the potential to significantly decrease traffic accidents caused by human error, thereby improving road safety. The environmental impact cannot be overlooked either; autonomous driving agents can optimize routes and driving patterns to reduce fuel consumption and emissions.

In conclusion, the development of knowledge-driven autonomous driving agents is a monumental step towards an autonomous future. It embodies the synergy between theoretical research and practical application, illustrating that even with limited resources, significant strides can be made in this cutting-edge field. As LLMs continue to evolve, their interpretability, expandability, and generalizability will only enhance the capabilities of autonomous driving agents, making the dream of driverless cars an imminent reality. The project not only contributes to the technological landscape but also has far-reaching effects on societal norms, safety, and the environment, marking a pivotal moment in the journey towards sustainable and intelligent transportation systems.
\section{\uppercase{FUTURE WORK}}
The future expansion of this work holds promising avenues for enhancing the robustness and applicability of the research. Firstly, the creation of more dataset data is crucial. This involves not only increasing the volume of data but also ensuring its diversity to cover a wide array of scenarios that the models may encounter. This could include varied environmental conditions, different geographic locations, and a multitude of operational contexts. 

Secondly, comparing newer and bigger models can provide insights into the scalability and efficiency of current algorithms. As computational power increases and new architectures are developed, it is imperative to assess their performance against established benchmarks. This comparison could lead to the discovery of more sophisticated models that offer improved accuracy and speed.

Enabling fine-tuning is another significant step. This process allows models to be more precisely adjusted to specific tasks by training on a smaller, task-specific dataset after being pre-trained on a large dataset. Fine-tuning can result in models that perform better on specialized tasks, adapting to the nuances of the data they are meant to interpret.

Testing in new maps is essential for validating the generalizability of the models. By exposing the models to previously unseen environments, researchers can evaluate their adaptability and identify areas where further training is needed. This helps in ensuring that the models are not just theoretically sound but also practically viable.

Lastly, rewriting the Dataset Collection tool to be a web app can significantly streamline the data collection process. A web-based tool can be more accessible to users, allowing for easier data input and management. It can also facilitate real-time data collection and analysis, which is invaluable for dynamic research environments.

Each of these steps is designed to build upon the existing foundation, pushing the boundaries of what is currently possible and paving the way for new discoveries in the field. The continuous evolution of technology necessitates an iterative approach to research, where each phase of expansion is seen as an opportunity to refine and enhance the work being done. The integration of these expansions will undoubtedly lead to more sophisticated and capable systems, driving progress in the field forward.
 \end{sloppypar}
 \end{spacing}